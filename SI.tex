%
%
%

\documentclass[journal=jctcce,manuscript=suppinfo]{achemso}

\usepackage[T1]{fontenc}
\usepackage{graphicx}
\usepackage{amsmath}
\usepackage{amssymb}
\usepackage{xcolor}


\title{Reproducing Relative Alchemical Free Energies of Hydration}

\author{Hannes H. Loeffler}
\affiliation[STFC Daresbury, Warrington, WA4 4AD, United
  Kingdom]{Scientific Computing Department, STFC, UK}
\email{Hannes.Loeffler@stfc.ac.uk}
\phone{+44 1925 603367}
\author{Stefano Bosisio}
\affiliation[University of Edinburgh]{EaStCHEM School of Chemistry,
  University of Edinburgh, West Mains Road, Edinburgh EH9 3JJ, UK}
\author{Julien Michel}
\affiliation[University of Edinburgh]{EaStCHEM School of Chemistry,
  University of Edinburgh, West Mains Road, Edinburgh EH9 3JJ, UK}
\author{Guilherme Duarte Ramos Matos}
\affiliation[University of California, Irvine]{Department of
  Chemistry, University of California, Irvine}
\author{David L. Mobley}
\affiliation[University of California, Irvine]{Departments of
  Pharmaceutical Sciences and Chemistry, University of California,
  Irvine}
\author{Donghyuk Suh}
\affiliation[University of Chicago]{University of Chicago}
\author{Benoit Roux}
\affiliation[University of Chicago]{University of Chicago}

%\abbreviations{$\Delta G^{solv}$, $\Delta H^{solv}$, $\Delta S^{solv}$}
\keywords{Free Energy, Hydration, Alchemical, Reproducibility}


\begin{document}

\maketitle

\section{Softcore Functions}

We describe here the softcore functions for both the van der Waals,
$V_{\mathrm{LJ}}$ (Lennard--Jones potential) and the electrostatic
interactions, $V_{\mathrm{Coul}}$ (Coulomb potential) for the
disappearing atoms as a function of the order parameter $\lambda$.
For the appearing atoms replace $\lambda$ with $1 - \lambda$ and
\emph{vice versa}.  Eq.\ \eqref{eq:general} is the generalized form
for all codes while the specific distance dependent functions are
outlined in eq.\,\eqref{eq:Sire} for Sire, eq.\,\eqref{eq:Amber} for
AMBER, eq.\,\eqref{eq:Gromacs} for Gromacs and eq.\,\eqref{eq:CHARMM}
for CHARMM.

\begin{equation}
  V = V_{\mathrm{LJ}} + V_{\mathrm{Coul}} =
  4\epsilon_{\mathrm{ij}}(1 - \lambda) \left[ \left(
      \frac{\sigma_{ij}}{\textcolor{red}{r_{\mathrm{LJ}}}}
    \right)^{12} - \left(
      \frac{\sigma_{ij}}{\textcolor{red}{r_{\mathrm{LJ}}}} \right)^{
      6} \right] +
  (1 - \lambda)^{n} \frac{q_{i}q_{j}}
  {4\pi\varepsilon_{0}\textcolor{red}{r_{\mathrm{Coul}}}}
  \label{eq:general}
\end{equation}

For Sire
\begin{equation}
  \begin{split}
    r_{\mathrm{LJ}} &= (\alpha\sigma_{ij}\lambda + r_{ij}^2)^{\frac{1}{2}} \\
    r_{\mathrm{Coul}} &=  (\lambda + r_{ij}^2)^{\frac{1}{2}}
  \end{split}
  \label{eq:Sire}
\end{equation}

For AMBER
\begin{equation}
  \begin{split}
    r_{\mathrm{LJ}} &= (\alpha \sigma_{ij}^{6} \lambda + % see JCP127, 214108
                         r_{ij}^6)^{\frac{1}{6}} \\
    r_{\mathrm{Coul}} &= (\beta\lambda + r_{ij}^{p})^{\frac{1}{p}} \\
    n &= 1
  \end{split}
  \label{eq:Amber}
\end{equation}

For Gromacs
\begin{equation}
  \begin{split}
    r_{\mathrm{LJ}} &= (\alpha \sigma_{ij}^{w} \lambda^{p} +
    r_{ij}^{w})^{\frac{1}{w}} \\
    &p = 1,2; w = 6,48; \\
    r_{\mathrm{Coul}} &= r_{\mathrm{LJ}} \\
    &\alpha_{\mathrm{Coul}} = 0,\alpha_{\mathrm{LJ}} \\
    n &= 1
  \end{split}
  \label{eq:Gromacs}
\end{equation}

For CHARMM (PSSP), applied to all ``reactant'' and all ``product'' atoms
\begin{equation}
  \begin{split}
    r_{\mathrm{LJ}} &= (\alpha \lambda + r_{ij}^2)^{\frac{1}{2}} \\
    r_{\mathrm{Coul}} &= (\beta\lambda + r_{ij}^{2})^{\frac{1}{2}} \\
    n &= 1
   \end{split}
  \label{eq:CHARMM}
\end{equation}

$r_{\mathrm{vdW}}$ and $r_{\mathrm{Coul}}$ (both in red) are
the distance dependent functions, $\epsilon_{\mathrm{Coul}}$ and
$\sigma_{ij}$ are the Lennard-Jones parameters, $q_{i}$ and $q_{j}$
are the charges and $\varepsilon_{0}$ is the vacuum permittivity,
$\alpha$ and $\beta$ are the softcore tuning parameters, and $r_{ij}$
the distance between atoms.

$n$ is an exponent only used in the Coulomb softcore funtion of Sire.
Gromacs allows additional exponents for $\lambda$ ($p = 1$ or $2$) and
$w$ for the distance dependency with values of either $6$ or $48$.
The Coulomb softcore parameter $\alpha_{\mathrm{Coul}}$ in Gromacs is
the same as for the Lennard--Jones parameter $\alpha_{\mathrm{LJ}}$
unless the Coulomb softcore function is requested not to be used.  The
CHARMM softcore function (PSSP) is applied to \emph{all} atoms in the
perturbed group and not only to disappearing/appearing atoms as in the
other codes.  The perturbed group comprises of all atoms that need to
be transformed, i.e.\ any atom that differs in at least one force
field parameter in the other end state.


\end{document}
